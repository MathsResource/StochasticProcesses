Some of the results mentioned above can be derived from properties of Pascal's triangle. The number of different walks of n steps where each step is +1 or −1 is 2n. For the simple random walk, each of these walks are equally likely. In order for Sn to be equal to a number k it is necessary and sufficient that the number of +1 in the walk exceeds those of −1 by k. The number of walks which satisfy S_n=k is equally the number of ways of choosing (n - k)/2 with n is the number of allowed moves,[12] denoted n \choose (n-k)/2. For this to have meaning, it is necessary that n and k be even numbers. Therefore, the probability that S_n=k is equal to 2^{-n}{n\choose (n-k)/2}. By representing entries of Pascal's triangle in terms of factorials and using Stirling's formula, one can obtain good estimates for these probabilities for large values of n.
