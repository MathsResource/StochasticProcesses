\documentclass{article}
\usepackage[utf8]{inputenc}
\usepackage{enumerate}
\title{MatrixTutorial1}
\author{kobriendublin }
\date{October 2017}

\begin{document}


\section{Brownian Motion (Wiener Process)}

Brownian motion is another widely-used random process. It has been used in engineering, finance, and physical sciences. It is a Gaussian random process and it has been used to model motion of particles suspended in a fluid, percentage changes in the stock prices, integrated white noise, etc. Figure 11.29 shows a sample path of Brownain motion. 

%----------IMAGE -------%

Figure 11.29 - A possible realization of Brownian motion.
In this section, we provide a very brief introduction to Brownian motion. It is worth noting that in order to have a deep understanding of Brownian motion, one needs to understand It$\hat{o}$ calculus, a topic that is beyond the scope of this book.

\subsection*{It$\hat{o}$ Calculus}
It$\hat{o}$ calculus, named after Kiyoshi It$\hat{o}$, extends the methods of calculus to stochastic processes such as Brownian motion (see Wiener process). It has important applications in mathematical finance and stochastic differential equations. 
The central concept is the Itô stochastic integral, a stochastic generalization of the Riemann–Stieltjes integral in analysis. The integrands and the integrators are now stochastic processes:

\[{\displaystyle Y_{t}=\int _{0}^{t}H_{s}\,dX_{s},} \]

where H is a locally square-integrable process adapted to the filtration generated by X (Revuz & Yor 1999, Chapter IV), which is a Brownian motion or, more generally, a semimartingale. The result of the integration is then another stochastic process. Concretely, the integral from 0 to any particular t is a random variable, defined as a limit of a certain sequence of random variables. The paths of Brownian motion fail to satisfy the requirements to be able to apply the standard techniques of calculus. So with the integrand a stochastic process, the Itô stochastic integral amounts to an integral with respect to a function which is not differentiable at any point and has infinite variation over every time interval. The main insight is that the integral can be defined as long as the integrand H is adapted, which loosely speaking means that its value at time t can only depend on information available up until this time. Roughly speaking, one chooses a sequence of partitions of the interval from 0 to t and construct Riemann sums. Every time we are computing a Riemann sum, we are using a particular instantiation of the integrator. It is crucial which point in each of the small intervals is used to compute the value of the function. The limit then is taken in probability as the mesh of the partition is going to zero. Numerous technical details have to be taken care of to show that this limit exists and is independent of the particular sequence of partitions. Typically, the left end of the interval is used. 

%============================================%

\end{document}
