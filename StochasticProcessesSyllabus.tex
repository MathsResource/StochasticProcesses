Stochastic Processes (MS4217)
Syllabus

Conditional probability and conditional expectations; Markov chains, Chapman-Kolmogorov equations, classification of states, limiting distributions, random walks, branching processes, time reversible Markov chains; Renewal Theory, counting processes, the Poisson process, semi-Markov processes; Queuing theory, the M/G/I and G/M/I systems, multiserver queues; continuous-time Markov chains, birth and death processes; Brownian motion with application in option pricing.
 
Chapman-Kolmogorov equations
The Chapman–Kolmogorov equation is an identity relating the joint probability distributions of different sets of coordinates on a stochastic process
 
Renewal Theory
 
Renewal theory is the branch of probability theory that generalizes Poisson processes for arbitrary holding times. Applications include calculating the expected time for a monkey who is randomly tapping at a keyboard to type the word Macbeth and comparing the long-term benefits of different insurance policies.
 
 
Probability Generating Function
    PGF of a binomial RV
 
    Abel's Theorem
        Abel's theorem for power series relates a limit of a power series to the sum of its coefficients.
 
Gambler's ruin
    Duration of the contest.

Birth-death processes
The birth-death process is a special case of continuous-time Markov process where the states represent the current size of a population and where the transitions are limited to births and deaths.
Birth-death processes have many applications in demography, queueing theory, performance engineering, or in biology, for example to study the evolution of bacteria.
 
