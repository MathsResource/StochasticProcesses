%---------------------------------------------------------------------%
%---------------------------------------------------------------------%

\chapter{Markov Chains}
\section{Markov Chains}

\[
T= \left[ \begin{array}{cccc} 0 & 0 & 0 & 0 \\ 0 & 0 & 0 & 0 \\0 & 0 & 0 & 0 \\0 & 0 & 0 & 0 \\
\end{array} \right]
\]

Find the eigenvalues of the following matrix.
\[
T= \left[ \begin{array}{ccc} 0.25 & 0.5 & 0.25  \\ 0.50 & 0.25 & 0.25   \\0.25 & 0.25 & 0.50   \\
\end{array} \right]
\]

\[
Det(T-I_3\lambda)= \left| \begin{array}{ccc} 0.25-\lambda & 0.5 & 0.25  \\ 0.50 & 0.25-\lambda & 0.25   \\0.25 & 0.25 & 0.50-\lambda   \\
\end{array} \right|
\]


\newpage



\section{Markov Chains}

\[
T= \left[ \begin{array}{cccc} p_{11} & p_{12} & \dots & p_{1m} \\
p_{21} & p_{22} & \dots & p_{2m} \\\vdots & \vdots & \ddots &
\vdots
\\ p_{m1} & p_{m2} & \dots & p_{mm} \\
\end{array} \right]
\]


\subsection{Classification of States}
\begin{enumerate}
	\item Absorbing state \item Periodic state \item Persistent state
	\item Transient state \item Ergodic state
\end{enumerate}


\subsection{Absorbing states}

Absorbing states are characterized in Markov chains by a value of
1 in the diagonal element of the matrix ($p_{ii} = 1$). Once
entered, there is no escaping the absorbing states.


\subsection{Classification of Chain}
\begin{enumerate}
	\item Irreducible sets \item Closed sets \item Ergodic chains
\end{enumerate}

\subsection{Irreducible Chain}
An irreducible chain is a chain in which every state is accessible
from any other state in a finite number of steps.

\subsection{Closed Sets}
A closed set is a subset of states that can not be escaped once
entered. An absorbing state is a closed set composed of one state.
\subsubsection{Closed Sets}
\[
T= \left[ \begin{array}{cccc} 0.2 & 0.2 & 0.3 & 0.3 \\ 0.1 & 0.2 & 0.3 & 0.4 \\0 & 0 & 0.5 & 0.5 \\0 & 0 & 0.5 & 0.5 \\
\end{array} \right]
\]

Once either states 3 or 4 are entered, it is not possible to
revert to states 1 or 2. States 3 and 4 will be the only states
visited.
\subsection{Ergodic Chains}
For Ergodic Chains, the invariant distribution is the vector of
the mean recurrence time reciprocals.

\newpage